\section{Construção da Heurística}
A menos que o espaço de estado seja pequeno o bastante para ser gerado
integralmente, considerando também o peso da prova de teoremas, o jogador deve
usar uma função heurística para avaliar os novos estados encontrados e, assim,
limitar a busca. Tal função de avaliação é específica de jogo para jogo. Contar
o número de peças a nosso favor é bom para um jogo como Damas, mas produz
resultados catastróficos em Resta Um. Grande parte do esforço despendido na
criação de um jogador tradicional, é feita por seres humanos tentando melhorar a
função de avaliação.

Baseada nas características ({\it features}) que podem ser extraídas da
descrição,CITE desenvolveram
um método para gerar heurísticas automaticamente. Nos atentaremos em como as
{\it features} são extraídas e, mais adiante, como os jogadores as usam para
construír as heurísticas.

O primeiro passo para a construção da heurística é reconhecer estruturas como
contadores, tabuleiros, peças e etc. Segundo CITE, os jogadores identificam as estruturas através de um
modelo, que mesmo sendo específicos para GDL, podem ser conceitualmente aplicada
a qualquer outra linguagem.

Como já mencionado acima, os tokens em GDL não tem nenhum siginificado semântico
para o jogador. Na competição os tokens são misturados para que somente a
sintatica seja considerada na construção dos jogadores. A relação sucessor, por
exemplo, pode ser descrita como:
\begin{verbatim}
   (foo on off) (foo off one) (foo one blue)
   (jap ichi ni) (jap ni san) (jap san shi)
\end{verbatim}
Ainda sim podemos identificar esta como sendo uma relação de sucessão, pois
seguem uma estrutura do tipo
\begin{verbatim}
   <succ> <atom 1> <atom 2>
   <succ> <atom 2> <atom 3>
\end{verbatim}
\hspace{2cm} \vdots
\begin{verbatim}
   <succ> <atom n-1> <atom n>   
\end{verbatim}
