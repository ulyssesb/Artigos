\section{Construção da Heurística}
A menos que o espaço de estado seja pequeno o bastante para ser gerado
integralmente, considerando também o peso da prova de teoremas, o jogador deve
usar uma função heurística para avaliar os novos estados encontrados e, assim,
limitar a busca. Tal função de avaliação é específica de jogo para jogo. Contar
o número de peças a nosso favor é bom para um jogo como Damas, mas produz
resultados catastróficos em Resta Um. Grande parte do esforço despendido na
criação de um jogador tradicional, é feita por seres humanos tentando melhorar a
função de avaliação.

Em GGP, a construção da heurística deve ser baseada apenas nas informações
contidas na descrição. O fator determinante para o sucesso do jogador está em
como utilizar as características ({\it features}) para compor a função de
avaliação. Cada jogador submetido à competição analisa a descrição e constroi a
função heurística de maneira ímpar. Veremos a seguir uma técnica simples para o
reconhecimento das {\it features} descrita por \cite{\dresner}.


\subsection{Identificando estruturas sintáticas}
Na descrição podemos reconhecer cinco estruturas básica: relações de sucessão,
contadores, tabuleiros, marcadores, peças e quantidades. A identificação destas
é feita comparando as relações, contidas na descrição, com modelos das
estruturas. Apesar dos modelos serem específicos para GDL, eles podem ser
conceitualmente adaptados para qualquer outra linguagem lógica.

Como mencionado anteriormente, não podemos depender dos tokens para a construção
do modelo. Na competição, a descrição dos jogos tem o nome dos tokens e relações
embaralhados para que o jogador não obtenha dicas da estrutura do jogo apenas
lendo relações como {\it cell}. A relação sucessor, por exemplo, poderia ser
reescrita como:
\begin{verbatim}
   (foo on off) (foo off one) (foo one blue)
   (jap ichi ni) (jap ni san) (jap san shi)
\end{verbatim}
Ainda sim podemos identificar esta como sendo uma relação de sucessão, pois
a sucessão é uma propriedade estrutural e não léxica.
\begin{verbatim}
   <succ> <atom 1> <atom 2>
   <succ> <atom 2> <atom 3>
\end{verbatim}
\hspace{2cm} \vdots
\begin{verbatim}
   <succ> <atom n-1> <atom n>   
\end{verbatim}
Sem a identificação desta relação, não conseguiríamos saber que a coluna {\it a}
esta próxima da coluna {\it b}, em um jogo como Xadrez ou Damas.

Um jogo pode conter várias realções de sucessão, que são usadas para identificar
{\it contadores}. Um contador é um termo funcional incrementado em cada estado
do jogo e podem ser identificados usando o seguinte modelo: 
\begin{verbatim}
   (<= (next (<counter> ?<var1>))
       (true (<counter> ?<var2>))
       (<successor> ?<var2> ?<var1>))
\end{verbatim}
onde $<$counter$>$ é a função identificada e $<$sucessor$>$ uma relação de sucessão. 

Outra estrutura identificavel é o {\it tabuleiro}. Um tabuleiro é uma matriz de
duas dimenções de células que mudam de valor. \cite{dresner} assume que toda
relação ternária de um estado é um tabuleiro. No entanto, um dos argumentos do
tabuleiro precisa corresponder a um estado da célula, que nunca pode ter dois
valores simultâneos. No jogo da velha temos um tabuleiro, a relação {\it cell}.

Após reconhecido o tabuleiro, podemos reconhecer os {\it marcadores}, que são
objetos que ocupam as células do tabuleiro. Se um marcador esta em apenas uma
célula de cada vez, então ele recebe o nome de {\it peça}. Em nosso exemplo, os
marcadores {\it x} e {\it o} não são considerados peças porque podem estar em
mais de uma célula do tabuleiro ao mesmo tempo.