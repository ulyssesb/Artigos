\section{Game Description Language}
Em GDL os jogos são tratados como máquinas de estado. A descrição de um jogo
consiste em um conjunto de fatos verdadeiros, que descrevem o estado atual, as
relações que modificam o estado atual, as regras do jogo, e qual o estado deve
ser atingido para o que o jogo acabe.
 
Apesar do objetivo do GGP englobar todos os tipos de jogos, a competição trata
apenas jogos {\it determinísticos} e {\it perfeitamente informados}. Se em um
jogo conseguirmos determinar seu próximo estado, dado o atual e as ações tomadas
por todos os jogadores, então ele é {\it determinístico}. Go e Otelo são
exemplos de jogos desta classe, já Gamão fica fora porque o próximo estado
depende de um fator não determinístico, os dados. Nos jogos {\it perfeitamente
  informados}
o estado do jogo é conhecido por todos os participantes. São
considerados perfeitamente informados xadrez e jogo da velha, pois o estado
atual está no campo de visão dos jogadores. Entretanto, truco e batalha naval,
onde os jogadores escondem para si parte do estado atual, não entram nesta
categoria. Jogos podem ter um ou mais jogadores, baseados em rodadas ou
simultâneos.
 
Um pequeno conjunto de palavras chaves é usado em GDL para criar a descrição dos
jogos: {\it role, init, next, true, does, terminal, goal} e {\it distinct}.
Como exemplo de construção na linguagem, usaremos o jogo da velha, onde os
estados correspondem a uma matriz 3x3 em que cada célula pode estar vazia, preenchida
com {\it x} ou {\it o}.
 
A palavra, ou {\it relação}, {\it role} é usada para descrever quais serão os
jogadores na partida. No jogo da velha temos a seguinte declaração:
\begin{verbatim}
  (role xplayer)
  (role oplayer)
\end{verbatim}
indicando dois jogadores, {\it x} e {\it o}.
 
O predicado {\it init} representa os fatos que são verdades no início do jogo.
No estado inicial do jogo da velha todos as células são vazias e quem começa é o
jogador {\it x}.
\begin{verbatim}
  (init (cell 1 1 b))
  (init (cell 1 2 b))
  (init (cell 1 3 b))
  (init (cell 2 1 b))
  (init (cell 2 2 b))
  (init (cell 2 3 b))
  (init (cell 3 1 b))
  (init (cell 3 2 b))
  (init (cell 3 3 b))
  (init (control xplayer))
\end{verbatim}
Os predicados {\it cell} e {\it control} são específicos em cada jogo. Os
números também não tem nenhum significado fora deste contexto. A mesma
declaração poderia ser feita da seguinte maneira:
\begin{verbatim}
  (init (booo foo foo beh))
  (init (booo foo bar beh))
  (init (booo foo xyz beh))
\end{verbatim}
\hspace{2cm} \vdots
\begin{verbatim}
  (init (trew xplayer))
\end{verbatim}
sem perder a semântica. Essa falta de comprometimento com a sintaxe tem um papel
importante na formulação da heurística, como veremos adiante.
 
As relações {\it true} e {\it init} tomam uma regra ou átomo como parâmetro, que
são verdadeiros no estado atual do jogo. A diferença entre eles é que {\it init}
é usada para criar o estado inicial do jogo e não é mais usada depois, enquanto
{\it true} fará as atualizações dos fatos nos demais estados. A declaração:

\begin{verbatim}
  (true (cell 2 2 b))
  (true (control xplayer))
\end{verbatim}
indica que no estado atual a célula do centro da matriz (posição 2,2) contém um
branco e o jogador {\it x} deverá jogar.
 
Análoga à {\it true}, a relação {\it next} refere aos fatos que serão verdades
no próximo estado. Em nosso exemplo, o controle das rodadas é feito da seguinte
maneira:
\begin{verbatim}
  (<= (next (control xplayer))
      (true (control oplayer)))
 
  (<= (next (control oplayer))
      (true (control xplayer)))
\end{verbatim}
O símbolo {\it $<$=} indica uma implicação reversa.
 
No jogo da velha um jogador poderá marcar uma das células se ela estiver vazia e
se for a sua vez de jogar. A declaração de quais jogadas podem ser feitas em um
determinado estado do jogo é descrita usando o axioma {\it legal}. Essa relação
toma como parâmetro um jogador e uma ação (ou movimento).
\begin{verbatim}
  (<= (legal ?player (mark ?x ?y))
      (true (cell ?x ?y b))
      (true (control ?player)))
 
  (<= (legal xplayer noop)
      (true (control oplayer)))
 
  (<= (legal oplayer noop)
      (true (control xplayer)))
\end{verbatim}
O axioma {\it noop} é usado no controle de rodadas quando não é a vez do
jogador, sua única ação é não fazer nada. Tokens iniciados com ``?'' são
variáveis.
 
 
Caso um jogador possa marcar uma célula, verificando através do {\it legal} se é
possível, no próximo estado do jogo é esperado que aquela célula contenha a
marca deixada. O resultado das ações legais tomadas é descrita usando a relação
{\it does}:
\begin{verbatim}
  (<= (next (cell ?m ?n x))
      (does xplayer (mark ?m ?n))
\end{verbatim}
No próximo estado a célula {\it ?m ?n} conterá um {\it x} se a jogada puder ser
efetuada pelo jogador {\it x} no atual estado.
 
Para o problema do quadro ({\it frame problem}) temos uma adição dos axiomas:
\begin{verbatim}
  (<= (next (cell ?x ?y b))
      (does ?player (mark ?m ?n))
      (true (cell ?x ?y b))
      (distinctCell ?x ?y ?m ?n))
 
  (<= (distinctCell ?x ?y ?m ?n)
      (distinct ?x ?m))
 
  (<= (distinctCell ?x ?y ?m ?n)
      (distinct ?y ?n))
\end{verbatim}
indicando que se uma célula contém um branco no estado atual e o jogador não a
marca, no próximo estado ela continuará contendo branco. A palavra chave {\it
  distinct}
é usada para comparar dois axiomas.
 
O fim de jogo é alcançado quando um dos jogadores consegue marcar uma sequência
de três células, em uma coluna, linha ou na diagonal, ou quando não há mais
espaços em branco.
\begin{verbatim}
  (<= terminal
      (line x))
 
  (<= terminal
      (line o))
 
  (<= terminal
      (not open))
\end{verbatim}
 
A pontuação que cada jogador consegue ao atingir o final do jogo é encontrada
nas regras {\it goal}.
\begin{verbatim}
  (<= (goal xplayer 100)
      (line x))
 
  (<= (goal xplayer 50)
      (not (line x))
      (not (line o))
      (not open))
 
  (<= (goal xplayer 0)
      (line o))
\end{verbatim}
O jogador {\it x} conseguirá 100 pontos se tiver uma sequência, 50 se der
``velha'' e nenhum se o adversário fizer a sequência, quando um dos estados
terminais for atingido. A descrição completa do jogo da velha pode ser
encontrada nos anexos.

A descrição de um jogo pode usar relações adicionais para simplificar as
condições de outras regras, como {\it line} no exemplo acima, e suportar
relações numéricas. No domínio do jogo da velha não se mostra necessário, porém
podemos definir: 
\begin{verbatim}
  (succ 1 2) (succ 2 3) (succ 3 4)
  (nextcol a b) (nextcol b c) (nextcol c d)
\end{verbatim}
A relação {\it succ} define como é incrementado algum contador, o número da
rodada em algum jogo que possui um limite de rodadas para acabar, por
exemplo. No xadrez, a localização das colunas seria feita usando a relação {\it
  nextcol}. Encontrar este tipo de relação númerica na descrição tem um grande 
valor, pois servem de ponte entre representação lógica e numérica.


Para que os jogos sejam {\it perfeitamente informados} as regras da competição
especificam que, antes de cada rodada, os jogadores recebem todas as jogadas
feitas no último turno.
\begin{verbatim}
  (does x (mark 2 2))
  (does o noop)
\end{verbatim}
Desta maneira é possível calcular em qual estado do jogo estamos.

\subsection{Prova de teoremas}
Para que seja possível derivar os movimentos legais, simular um jogo e
determinal o final do mesmo, um jogador deve utilizar um provador de
teoremas. Com a declaração do estado inicial, e as atualizações dos movimentos,
podemos inferir quais jogadas podemos realizar no estado atual. Para ilustrar,
tomemos como exemplo o seguinte estado parcial:
\begin{enumerate}
  \item \verb|cell(1 1 b)|
  \item \verb|control(xplayer)|
\end{enumerate}

Os movimentos legais e seus efeitos na base de conhecimento podem ser traduzidos
como implicações. 
\begin{enumerate}
  \setcounter{enumi}{2}
  \item \verb|(cell(?x ?y b)| $\land$ \verb|control(?player))|  
    $\Rightarrow$ \verb|legal(?player (mark ?x ?y))|
  \item \verb|legal(xplayer (mark ?x ?y)| $\Rightarrow$ 
    \verb|next(cell(?x ?y x))|
\end{enumerate}

Usando o provador de teoremas, podemos inferir 
\begin{enumerate}
  \setcounter{enumi}{4}
  \item \verb|legal(xplayer (mark 1 1))|, por {\it 1, 2} e {\it 3}
  \item \verb|next(cell(1 1 x))|, por {\it 4} e {\it 5}
\end{enumerate}

Assim conseguimos descobrir que é possivel no estado atual marcar a primeira
célula (posição 1,1) do tabuleiro. Da mesma maneira, outras jogadas possíveis
são inferidas. No entanto, a prova de teoremas é custosa computacionalmente, o
que agrava o problema da construção da árvore de buscas, aumentando ainda mais o
valor de uma boa heurística.
