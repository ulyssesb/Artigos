\section{Introdução}
A ideia de construir um programa de computador capaz de derrotar um ser humano 
em um jogo complexo, xadrez por exemplo, foi alcançada a tempos, porém este 
resultado não mostra que a máquina tenha mais inteligência que o jogador. O 
DeepBlue, computador responsável pela vitória sobre o mestre de xadrez Kasparov, 
foi projeto especificamente para jogar um único jogo, se o confronto fosse um 
jogo da velha, ou mesmo alguma variante do xadrez, a máquina mal saberia 
responder uma jogada válida. 

Em contrapartida aos robos desenvolvidos para jogos específicos, existe uma ramo
da Inteligência Aritificial que aplica conhecimentos de lógica, busca 
heurística, representação do conhecimento, entre outros, para a criações de
jogadores que possam jogar qualquer tipo de jogo. 
O desafio dos Jogos Generalistas ({\it General Game Playing}, {\bf GGP}) é
contruir um programa que possa interagir e, principalmente, ganhar, sem qualquer
conhecimento prévio da mecânica do jogo. O programa deve raciocinar apenas sobre
a descrição do jogo que lhe é dada. Para promover o avanço nesta área, a
Universidade de Stanford organiza, desde 2005, uma competição anual entre
jogadores generalistas. 

Este artigo apresentará como pode ser feita a construção de um jogador, a
linguagem usada para descrever os jogos e quais são as características dos
jogadores submetidos à competição. 
