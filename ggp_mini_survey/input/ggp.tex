\section{General Game Playing}
Jogadores tradicionais como DeepBlue\cite{dblue}, Chinok\cite{chinok} e
TD-Gammon\cite{tesauro} usam árvores como estrutura computacional para
representar e analisar as jogadas (ou nós). Isto só é possível porque na
construção do jogador, o programador tem o conhecimento prévio das regras, logo,
sabe quais jogadas podem ser feitas, independente se são boas ou ruins. No
entanto, na maioria dos jogos, o espaço de estados para busca é muito grande,
tornando inviável percorrer todos os nós até achar um caminho que leve a vitória
(busca exaustiva). 

A solução é usar uma função que avalie quais nós são mais promissores, permitindo
que os piores não sejam expandidos. Esta função, chamada de heurística, é gerada
à partir da análise sobre as regras e de experiência adquirida durante os
jogos. No jogo de damas, por exemplo, uma heurística plausível seria a diferença
do número de peças em relação ao adversário. Utilizando esta heurística, pode-se
comparar dois nós e decidir qual caminho tomar na árvore.

O cenário proposto pelo GGP inviabiliza as técnicas clássicas na construção de
jogadores porque elas são específicas para cada jogo. Nos jogos generalistas,
não sabemos de antemão qual jogo o programa irá enfrentar, logo a ávore de busca
só será criada depois que receber quais são as regras e delas tirar os movimentos
válidos. Também não podemos contar com experiência no jogo, para derivar uma
heuristica. Esta é construída com técnicas que diferem de jogador para jogador. 

A descrição dos jogos na competição é feita através de uma linguagem derivada da
lógica de primeira ordem, {\it Game Description Language} ({\bf GDL}). As
decisões do jogador serão baseadas nas informações que podemos retirar e inferir
desta descrição.  
