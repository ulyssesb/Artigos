\documentclass[a4paper]{sbgames}               % final
%\usepackage[scaled=.92]{helvet}
\usepackage{times}
\usepackage{graphicx}

\usepackage[brazil]{babel}   
\usepackage[utf8]{inputenc}  

%% use this for zero \parindent and non-zero \parskip, intelligently.
\usepackage{parskip}

%% the 'caption' package provides a nicer-looking replacement
\usepackage[labelfont=bf,textfont=it]{caption}

\usepackage{url}

%% Paper title.
\title{Uma visão geral sobre General Game Playing}

%% Author and Affiliation (multiple authors). Use: and between authors

\author{Ulysses Bonfim}
\contactinfo{ubds06@c3sl.ufpr.br}
\affiliation{Universidade Federal do Paraná}

%% Keywords that describe your work.
\keywords{General Game Playing, GGP Competition}

%% Start of the paper
% Attention: As you need to insert EPS images in Postscript, 
% you need to insert PDF images into PDFs. 
% In the text, extensions cancbe omitted (latex use .eps, pdflatex get .pdf) 
% To convert them: epstopdf myimage.eps
\begin{document}

%\teaser{
%  \includegraphics[width=\linewidth]{sample.pdf}
%  \caption{Optional image}
%}

%% The ``\maketitle'' command must be the first command after the
%% ``\begin{document}'' command. It prepares and prints the title block.

\maketitle

%% Sections
\begin{abstract}
Resumo
\end{abstract}

%% The ``\keywordlist'' command prints out the keywords.
\keywordlist
\contactlist

\section{Introdução}
A ideia de construir um programa de computador capaz de derrotar um ser humano 
em um jogo complexo, xadrez por exemplo, foi alcançada a tempos, porém este 
resultado não mostra que a máquina tenha mais inteligência que o jogador. O 
DeepBlue, computador responsável pela vitória sobre o mestre de xadrez Kasparov, 
foi projetado especificamente para jogar um único jogo, se o confronto fosse um 
jogo da velha, ou mesmo alguma variante do xadrez, a máquina mal saberia 
responder uma jogada válida. 

Em contrapartida aos robos desenvolvidos para jogos específicos, existe um ramo
da Inteligência Aritificial que aplica conhecimentos de lógica, busca 
heurística, representação do conhecimento, entre outros, para a criações de
jogadores que possam jogar qualquer tipo de jogo. 
O desafio dos Jogos Generalistas ({\it General Game Playing}, {\bf GGP}) é
contruir um programa que possa interagir e, principalmente, ganhar, sem qualquer
conhecimento prévio da mecânica do jogo. O programa deve raciocinar apenas sobre
a descrição do jogo que lhe é dada. Para promover o avanço nesta área, a
Universidade de Stanford organiza, desde 2005, uma competição anual entre
jogadores generalistas. 

Este artigo apresentará como pode ser feita a construção de um jogador, a
linguagem usada para descrever os jogos e quais são as características dos
jogadores submetidos à competição. 

\section{General Game Playing}
Jogadores tradicionais como DeepBlue\cite{dblue}, Chinok\cite{chinok} e
TD-Gammon\cite{tesauro} usam árvores como estrutura computacional para
representar e analisar as jogadas (ou nós). Isto só é possível porque na
construção do jogador, o programador tem o conhecimento prévio das regras, logo,
sabe quais jogadas podem ser feitas, independente se são boas ou ruins. No
entanto, na maioria dos jogos, o espaço de estados para busca é muito grande,
tornando inviável percorrer todos os nós até achar um caminho que leve a vitória
(busca exaustiva). 

A solução é usar uma função que avalie quais nós são mais promissores,permitindo
que os piores não sejam expandidos. Esta função, chamada de heurística, é gerada
à partir da análise sobre as regras e de experiência adquirida durante os
jogos. No jogo de damas, por exemplo, uma heurística plauzível seria a diferença
do número de peças em relação ao adversário. Utilizando esta heurística, pode-se
comparar dois nós e decidir qual caminho tomar na árvore.

O cenário proposto pelo GGP inviabiliza as técnicas clássicas na construção de
jogadores porque elas são específicas para cada jogo. Nos jogos generalistas,
não sabemos de antemão qual jogo o programa irá enfrentar, logo a ávore de busca
só será criada depois que receber quais são as regras e delas tirar os movimentos
válidos. Também não podemos contar com experiência no jogo, para derivar uma
heuristica. Esta é construída com técnicas que diferem de jogador para jogador. 

A descrição dos jogos na competição é feita através de uma linguagem derivada da
lógica de primeira ordem, {\it Game Description Language} ({\bf GDL}). As
decisões do jogador serão baseadas nas informações que podemos retirar e inferir
desta descrição.  
\section{Game Description Language}
Em GDL os jogos são tratados como máquinas de estado. A descrição de um jogo
consiste em um conjunto de fatos verdadeiros, que descrevem o estado atual, as
relações que modificam o estado atual, as regras do jogo, e qual o estado deve
ser atingido para o que o jogo acabe.
 
Apesar do objetivo do GGP englobar todos os tipos de jogos, a competição trata
apenas jogos {\it determinísticos} e {\it perfeitamente informados}. Se em um
jogo conseguirmos determinar seu próximo estado, dado o atual e as ações tomadas
por todos os jogadores, então ele é {\it determinístico}. Go e Otelo são
exemplos de jogos desta classe, já Gamão fica fora porque o próximo estado
depende de um fator não determinístico, os dados. Nos jogos {\it perfeitamente
  informados}
o estado do jogo é conhecido por todos os participantes. São
considerados perfeitamente informados xadrez e jogo da velha, pois o estado
atual está no campo de visão dos jogadores. Entretanto, truco e batalha naval,
onde os jogadores escondem para si partedo estado atual, não entram nesta
categoria. Jogos podem ter um ou mais jogadores, baseados em rodadas ou
simultâneos.
 
Um pequeno conjunto de palavras chaves é usado em GDL para criar a descrição dos
jogos: {\it role, init, next, true, does, terminal, goal} e {\it distinct}.
Como exemplo de construção na linguagem, usaremos o jogo da velha, onde os
estados correspondem a uma matriz 3x3 em que célula pode estar vazia, preenchida
com {\it x} ou {\it o}.
 
A palavra, ou {\it relação}, {\it role} é usada para descrever quais serão os
jogadores na partida. No jogo da velha temos a seguinte declaração:
\begin{verbatim}
  (role xplayer)
  (role oplayer)
\end{verbatim}
indicando dois jogadores, {\it x} e {\it o}.
 
O predicado {\it init} representa os fatos que são verdades no início do jogo.
No estado inicial do jogo da velha todos as células são vazias e quem começa é o
jogador {\it x}.
\begin{verbatim}
  (init (cell 1 1 b))
  (init (cell 1 2 b))
  (init (cell 1 3 b))
  (init (cell 2 1 b))
  (init (cell 2 2 b))
  (init (cell 2 3 b))
  (init (cell 3 1 b))
  (init (cell 3 2 b))
  (init (cell 3 3 b))
  (init (control xplayer))
\end{verbatim}
Os predicados {\it cell} e {\it control} são específicos em cada jogo. Os
números também não tem nenhum significado fora deste contexto. A mesma
declaração poderia ser feita da seguinte maneira:
\begin{verbatim}
  (init (booo foo foo beh))
  (init (booo foo bar beh))
  (init (booo foo xyz beh))
\end{verbatim}
\hspace{2cm} \vdots
\begin{verbatim}
  (init (trew xplayer))
\end{verbatim}
sem perder a semântica. Essa falta de comprometimento com a sintaxe tem um papel
importante na formulação da heurística, como veremos adiante.
 
As relações {\it true} e {\it init} tomam uma regra ou átomo como parâmetro, que
são verdadeiros no estado atual do jogo. A diferença entre eles é que {\it init}
é usada para criar o estado inicial do jogo e não é mais usada depois, enquanto
{\it true} fará as atualizações dos fatos nos demais estados. A declaração:

\begin{verbatim}
  (true (cell 2 2 b))
  (true (control xplayer))
\end{verbatim}
indica que no estado atual a célula do centro da matriz (posição 2,2) contém um
branco e o jogador {\it x} deverá jogar.
 
Análoga à {\it true}, a relação {\it next} refere aos fatos que serão verdades
no próximo estado. Em nosso exemplo, o controle das rodadas é feito da seguinte
maneira:
\begin{verbatim}
  (<= (next (control xplayer))
      (true (control oplayer)))
 
  (<= (next (control oplayer))
      (true (control xplayer)))
\end{verbatim}
O símbolo {\it $<$=} indica uma implicação reversa.
 
No jogo da velha um jogador poderá marcar uma das células se ela estiver vazia e
se for a sua vez de jogar. A declaração de quais jogadas podem ser feitas em um
determinado estado do jogo é descrita usando o axioma {\it legal}. Essa relação
toma como parâmetro um jogador e uma ação (ou movimento).
\begin{verbatim}
  (<= (legal ?player (mark ?x ?y))
      (true (cell ?x ?y b))
      (true (control ?player)))
 
  (<= (legal xplayer noop)
      (true (control oplayer)))
 
  (<= (legal oplayer noop)
      (true (control xplayer)))
\end{verbatim}
O axioma {\it noop} é usado no controle de rodadas quando não é a vez do
jogador, sua única ação é não fazer nada. Tokens iniciados com ``?'' são
variáveis.
 
 
Caso um jogador possa marcar uma célula, verificando através do {\it legal} se é
possível, no próximo estado do jogo é esperado que aquela célula contenha a
marca deixada. O resultado das ações legais tomadas é descrita usando a relação
{\it does}:
\begin{verbatim}
  (<= (next (cell ?m ?n x))
      (does xplayer (mark ?m ?n))
\end{verbatim}
No próximo estado a célula {\it ?m ?n} conterá um {\it x} se a jogada puder ser
efetuada pelo jogador {\it x} no atual estado.
 
Para o problema do quadro ({\it frame problem}) temos uma adição dos axiomas:
\begin{verbatim}
  (<= (next (cell ?x ?y b))
      (does ?player (mark ?m ?n))
      (true (cell ?x ?y b))
      (distinctCell ?x ?y ?m ?n))
 
  (<= (distinctCell ?x ?y ?m ?n)
      (distinct ?x ?m))
 
  (<= (distinctCell ?x ?y ?m ?n)
      (distinct ?y ?n))
\end{verbatim}
indicando que se uma célula contém um branco no estado atual e o jogador não a
marca, no próximo estado ela continuará contendo branco. A palavra chave {\it
  distinct}
é usada para comparar dois axiomas.
 
O fim de jogo é alcançado quando um dos jogadores consegue marcar uma sequência
de três células, em uma coluna, linha ou na diagonal, ou quando não há mais
espaços em branco.
\begin{verbatim}
  (<= terminal
      (line x))
 
  (<= terminal
      (line o))
 
  (<= terminal
      (not open))
\end{verbatim}
 
A pontuação que cada jogador consegue ao atingir o final do jogo é encontrada
nas regras {\it goal}.
\begin{verbatim}
  (<= (goal xplayer 100)
      (line x))
 
  (<= (goal xplayer 50)
      (not (line x))
      (not (line o))
      (not open))
 
  (<= (goal xplayer 0)
      (line o))
\end{verbatim}
O jogador {\it x} conseguirá 100 pontos se tiver uma sequência, 50 se der
``velha'' e nenhum se o adversário fizer a sequência, quando um dos estados
terminais for atingido. A descrição completa do jogo da velha pode ser
encontrada nos anexos.

A descrição de um jogo pode usar relações adicionais para simplificar as
condições de outras regras, como {\it line} no exemplo acima, e suportar
relações numéricas. No domínio do jogo da velha não se mostra necessário, porém
podemos definir: 
\begin{verbatim}
  (succ 1 2) (succ 2 3) (succ 3 4)
  (nextcol a b) (nextcol b c) (nextcol c d)
\end{verbatim}
A relação {\it succ} define como é incrementado algum contador, o número da
rodada em algum jogo que possui um limite de rodadas para acabar, por
exemplo. No xadrez, a localização das colunas seria feita usando a relação {\it
  nextcol}. Encontrar este tipo de relação númerica na descrição tem um grande 
valor, pois servem de ponte entre representação lógica e numérica.


Para que os jogos sejam {\it perfeitamente informados} as regras da competição
especificam que, antes de cada rodada, os jogadores recebem todas as jogadas
feitas no último turno.
\begin{verbatim}
  (does x (mark 2 2))
  (does o noop)
\end{verbatim}
Desta maneira é possível calcular em qual estado do jogo estamos.

\subsection{Prova de teoremas}
Para que seja possível derivar os movimentos legais, simular um jogo e
determinal o final do mesmo, um jogador deve utilizar um provador de
teoremas. Com a declaração do estado inicial, e as atualizações dos movimentos,
podemos inferir quais jogadas podemos realizar no estado atual. Para ilustrar,
tomemos como exemplo o seguinte estado parcial:
\begin{enumerate}
  \item \verb|cell(1 1 b)|
  \item \verb|control(xplayer)|
\end{enumerate}

Os movimentos legais e seus efeitos na base de conhecimento podem ser traduzidos
como implicações. 
\begin{enumerate}
  \setcounter{enumi}{2}
  \item \verb|(cell(?x ?y b)| $\land$ \verb|control(?player))|  
    $\Rightarrow$ \verb|legal(?player (mark ?x ?y))|
  \item \verb|legal(xplayer (mark ?x ?y)| $\Rightarrow$ 
    \verb|next(cell(?x ?y x))|
\end{enumerate}

Usando o provador de teoremas, podemos inferir 
\begin{enumerate}
  \setcounter{enumi}{4}
  \item \verb|legal(xplayer (mark 1 1))|, por {\it 1, 2} e {\it 3}
  \item \verb|next(cell(1 1 x))|, por {\it 4} e {\it 5}
\end{enumerate}

Assim conseguimos descobrir que é possivel no estado atual marcar a primeira
célula (posição 1,1) do tabuleiro. Da mesma maneira, outras jogadas possíveis
são inferidas. No entanto, a prova de teoremas é custosa computacionalmente, o
que agrava o problema da construção da árvore de buscas, aumentando ainda mais o
valor de uma boa heurística.
\section{Construção da Heurística}
A menos que o espaço de estado seja pequeno o bastante para ser gerado
integralmente, considerando também o peso da prova de teoremas, o jogador deve
usar uma função heurística para avaliar os novos estados encontrados e, assim,
limitar a busca. Tal função de avaliação é específica de jogo para jogo. Contar
o número de peças a nosso favor é bom para um jogo como Damas, mas produz
resultados catastróficos em Resta Um. Grande parte do esforço despendido na
criação de um jogador tradicional, é feita por seres humanos tentando melhorar a
função de avaliação.

%DIEGO
% o que é CITE? É algo no plural?
Baseada nas características ({\it features}) que podem ser extraídas da
descrição, CITE desenvolveram
um método para gerar heurísticas automaticamente. Nos atentaremos em como as
{\it features} são extraídas e, mais adiante, como os jogadores as usam para
construír as heurísticas.

%DIEGO
% Achei confuso o final parágrafo, talvez algum problema com o tempo verbal,
% ou sujeito da frase (nem sei do que eu to falando).
O primeiro passo para a construção da heurística é reconhecer estruturas como
contadores, tabuleiros, peças e etc. Segundo CITE, os jogadores identificam as estruturas através de um
modelo, que mesmo sendo específicos para GDL, podem ser conceitualmente aplicada
a qualquer outra linguagem.

%DIEGO
% é sintatica mesmo?
% o que é um token em GDL?
Como já mencionado acima, os tokens em GDL não tem nenhum siginificado semântico
para o jogador. Na competição os tokens são misturados para que somente a
sintatica seja considerada na construção dos jogadores. A relação sucessor, por
exemplo, pode ser descrita como:
\begin{verbatim}
   (foo on off) (foo off one) (foo one blue)
   (jap ichi ni) (jap ni san) (jap san shi)
\end{verbatim}
Ainda sim podemos identificar esta como sendo uma relação de sucessão, pois
seguem uma estrutura do tipo
\begin{verbatim}
   <succ> <atom 1> <atom 2>
   <succ> <atom 2> <atom 3>
\end{verbatim}
\hspace{2cm} \vdots
\begin{verbatim}
   <succ> <atom n-1> <atom n>   
\end{verbatim}

\section{Conclusão}



\bibliographystyle{sbgames}
\bibliography{main}
\end{document}
